\documentclass{beamer}
\usetheme{Boadilla}
%==================================================
%Mypackages
\usepackage{epsf, amsmath, amssymb, epsfig, amsthm}
\usepackage{subfigure} % For subfigures
\usepackage{setspace} 
\usepackage{fancyhdr} 
\usepackage{eurosym}  %To write a Euro symbol
\usepackage[euler]{textgreek}
\usepackage[main=english,russian,spanish]{babel}
\usepackage[utf8]{inputenc}
\usepackage{graphicx}
\usepackage{float}
\usepackage{bbm}
\usepackage{cite}
\usepackage{listings}
\usepackage[font=small,labelfont=bf]{caption}
\usepackage{algorithm} 
\usepackage{algpseudocode} 
\lstset{
  basicstyle=\ttfamily,
  columns=fullflexible,
  frame=single,
  breaklines=true,
}
\hypersetup{colorlinks,urlcolor=blue}
\captionsetup[figure]{font=small}
\setbeamerfont{caption}{size=tiny}

%==================================================
%My commands: Define your commands here:

\begin{document}
\title{Planes, Trains, and Afflictions}
\subtitle{Agent-based modeling for the spread of disease through transportation networks}
\author[NS Group 3]{Frank Acquaye\\
Ho Lum Cheung\\
Dimas Muñoz-Montesinos\\
Elie Wanko}
\institute[HSE]
{
  \inst{}%
  Faculty of Computer Science\\
  Higher School of Economics\\
  Moscow
}
\date{27 JUN 2020}
\begin{frame}
\titlepage
\end{frame}
\begin{frame}
\frametitle{Table of Contents}
\begin{itemize}
    \item Motivation
    \item Prior Research
    \item ABM Framework
    \item Examples
    \item Deep Dive: New York City (NYC) Subway
    \item Discussion
    \item Conclusion
    \item Credits
\end{itemize}
\end{frame}
\begin{frame}
\frametitle{Motivation}
Why were certain areas of NYC more affected by COVID-19?\\
How were the subways involved?\\
\begin{figure}
\includegraphics[width=4cm]{Scratch_Visuals/NYC_COVID_Subways_Jeff.png}%
\includegraphics[width=4cm]{Scratch_Visuals/NYC_COVID_Case_Rate.png}
\caption{
Left: Case Rate And Subway On April 12, 2020 \cite{subway_epidemic_seed}\\
Right: NYC COVID Case Rate As Of June 23, 2020 \cite{nyc_health_covid_summary}
}
\end{figure}
\end{frame}
\begin{frame}
\frametitle{Prior Research (London Underground)}
Data: Oyster (Card), CASA (Timetable), PHE Data (Demographics)\\
\includegraphics[width=0.5\textwidth]{Scratch_Visuals/London_Walking_time.png}%
\includegraphics[width=0.5\textwidth]{Scratch_Visuals/London_Borough_Comparison.png}
\textit{Results show a correlation between the use of the underground and ILI cases in London, specifically they show that higher numbers of ILI cases arise in those boroughs where the population spend more time in the Underground and/or incur in a higher number of contacts when travelling.}
 \cite{gosce_johansson_2018}
\end{frame}
\begin{frame}
\frametitle{COVID in London}
\includegraphics[width=0.7\textwidth]{Scratch_Visuals/London_Covid_Adapted.png}
\end{frame}
\begin{frame}
\frametitle{Prior Research (Singapore Buses)}
\includegraphics[width=0.5\textwidth]{Scratch_Visuals/Singapore_1.png}%
\includegraphics[width=0.5\textwidth]{Scratch_Visuals/Singapore_2.png}
\textit{The direct contact in trains is, however, difficult to obtain from smart card data because the transactions are recorded at the station level}
\cite{singapore_bus_2020}
\end{frame}
\begin{frame}
\frametitle{ABM Framework}
\includegraphics[width=1.0\textwidth]{Scratch_Visuals/covid_framework_summary.png}
Components: Python, MESA, Networkx\\
Base Classes: \href{https://github.com/cheung-ho-lum/NS_Epidemics_ABM_Approach/blob/master/Report/Design_doc_for_expansion_of_subway_model.pdf}{Transportation Model, SEIR Agent}
\end{frame}
\begin{frame}{Embedded Animation}
\frametitle{Simple Geometries}
\includegraphics[width=1.0\textwidth]{Scratch_Visuals/geometries_example.png}
\begin{itemize}
    \item \href{https://github.com/cheung-ho-lum/NS_Epidemics_ABM_Approach/blob/master/Repository/Visualizations/infection_timelapse_grid_01.gif}{Grid}
    \item \href{https://github.com/cheung-ho-lum/NS_Epidemics_ABM_Approach/blob/master/Repository/Visualizations/infection_timelapse_sierpinski_01.gif}{Sierpinski's Triangle}
    \item \href{https://github.com/cheung-ho-lum/NS_Epidemics_ABM_Approach/blob/master/Repository/Visualizations/infection_timelapse_moscow.gif}{Moscow}
\end{itemize}
\end{frame}
\begin{frame}
\frametitle{World Airline Network (Passenger Flow)}
\begin{itemize}
    \item Investigate similarities and differences with subways
    \item Make framework more robust
\end{itemize}
\includegraphics[width=1.0\textwidth]{Scratch_Visuals/covid_airlines_initial.png}
\end{frame}
\begin{frame}
\frametitle{World Airline Network (Passenger Flow)}
Main Idea: Detailed route data can help model spread of diseases.
\begin{itemize}
    \item Regional / Domestic / EU Domestic / International Airports
    \item Regional / Domestic / EU Domestic / International Air Routes
\end{itemize}
Problem: This data not widely available. Solutions \cite{mao_wu_huang_tatem_2015}:
\begin{enumerate}
	\item Commercial Providers
	\item Open Data
	\item Statistical Modeling
\end{enumerate}
\end{frame}
\begin{frame}
\frametitle{World Airline Network (Passenger Flow)}
\begin{itemize}
    \item List of Airports - OpenFlights \cite{openflights}
    \item List of Air Routes - OpenFlights \cite{openflights}
    \item Airport Passenger Flow Numbers - \cite{airport_traffic}
    \item Top Air Routes \cite{airport_pairs}
    \item Other Air Routes
\end{itemize}
\begin{enumerate}
    \item Process airports, air routes
    \item Add all known passenger flow numbers
	\item Add all known airport and air route labels
	\item Classify remaining airports and air routes
	\item Interpolate missing air route data 
\end{enumerate}
\end{frame}
\begin{frame}[shrink=5]
\frametitle{World Airline Network Visualization (\href{https://github.com/cheung-ho-lum/NS_Epidemics_ABM_Approach/blob/master/Repository/Visualizations/infection_timelapse_world.gif}{gif})}
\includegraphics[width=1.0\textwidth]{Scratch_Visuals/covid_airlines.png}
\end{frame}
\begin{frame}
\frametitle{Madrid Commuter Trains (Central Hubs)}
How important are central transportation hubs to disease spread?
Data Source (madrid renfe gtfs)
\end{frame}
\begin{frame}
\frametitle{Madrid Commuter Trains (Central Hubs)}
Methods (ABM)
\end{frame}
\begin{frame}
\frametitle{Results (Madrid)}
\end{frame}
\begin{frame}
\frametitle{NYC Demographics And COVID Timeline
\includegraphics[width=1cm]{Scratch_Visuals/NYC_COVID_Case_Rate.png}
}
\textit{NYC residents working in Manhattan primarily travel by subway. This is
also true for residents of the Bronx, Brooklyn, and Queens\cite{nyc_commuting}}
\begin{itemize}
	\item March 9 - Mayor holds press conference and notes that there have been 16 confirmed cases. (106 cases)
	\item March 12 - Mayor declares a local state of emergency. (687 cases)
	\item March 15 - Schools officially close. (2,986 cases)
\end{itemize}
Borough - A geographical region. NYC has 5 boroughs. \\
MODZCTA - Modified Zip Code Tabulation Areas. $\propto$ postal codes.\\
\end{frame}
\begin{frame}
\frametitle{Subway Systems}
Complex, Station, Line, Route
\begin{figure}
    \includegraphics[width=6cm]{Scratch_Visuals/Novokuznetskaya.png}
    \caption{Floor Plan of Novokuznetskaya Metro Station \cite{Novokuznetskaya}}
\end{figure}
\end{frame}
\begin{frame}[fragile, shrink=5]
\frametitle{NYC Subway Data}
Stations, Map, Turnstiles (GTFS not considered)
\begin{figure}
    \includegraphics[width=6cm]{Scratch_Visuals/schneider_ridership.png}
    \caption{NYC Subway daily turnstile entries for March 2020 \cite{toddwschneider}}
\end{figure}
\begin{verbatim}
167,167,A32,IND,8th Av - Fulton St,W 4 St,M,A C E...
167,167,D20,IND,6th Av - Culver,W 4 St,M,B D F M...
\end{verbatim}
\end{frame}
\begin{frame}[shrink=5]
\frametitle{NYC Subway Modeling}
\begin{algorithm}[H]
\caption{Simulation of Disease Spread on Subways}\label{euclid}
\begin{algorithmic}[1]
\For{$i=1; i < TIMESPAN; i++$}
    \State Check conditions (i, number of infected) to see if we should deploy COUNTERMEASURES
    \For{Station in SubwayModel.Environment.Nodes}
        \State Calculate 'Local Exposure' from infected and commute time.
        \State Calculate 'Route Exposure' from infected on the same route.
        \State Calculate 'General Exposure' due to city-wide infected.
        \State Update 'Exposure' at station based on above conditions
    \EndFor
    \For{Agent in SubwayNetwork.Agents}
        \State Get 'Exposure' At Location
        \State Get City-wide COUNTERMEASURES
        \State Get Percentage of commuters
        \State Calculate SEIR beta and gamma based on conditions
        \State Update SEIR numbers
    \EndFor
\EndFor
\end{algorithmic}
\end{algorithm}
Subway Agent - SEIR Agent with additional exposure based on location. \\
Subway Graph - Nodes $\propto$ Stations, Edges $\propto$ \textbf{Lines, Complex}, Route Lookup.
\end{frame}
\begin{frame}
\frametitle{NYC Model Parameters, Hyper-parameters}
\begin{table}[htbp]
\begin{tabular}{|l|l|l|l|l|}
\hline
\textbf{} &
  \textbf{\begin{tabular}[c]{@{}l@{}}Contact\\ Rate ($\beta$)\end{tabular}} &
  \textbf{\begin{tabular}[c]{@{}l@{}}Latent\\ Rate ($\alpha$)\end{tabular}} &
  \textbf{\begin{tabular}[c]{@{}l@{}}Removal\\ Rate ($\gamma$)\end{tabular}} &
  \textbf{\begin{tabular}[c]{@{}l@{}}Start\\ Trigger\end{tabular}} \\ \hline
\textbf{Default}                                                           & \textbf{1.75} & \textbf{0.2} & \textbf{0.5} & t = 0               \\ \hline
\textbf{\begin{tabular}[c]{@{}l@{}}Isolation\\ Modifier\end{tabular}}      & 0.25          & 1            & 2            & I \textgreater 5000 \\ \hline
\textbf{\begin{tabular}[c]{@{}l@{}}Recommendation\\ Modifier\end{tabular}} & 0.67          & 1            & 1.5          & I \textgreater 500  \\ \hline
\textbf{\begin{tabular}[c]{@{}l@{}}Awareness\\ Modifier\end{tabular}}      & 1 to 0.25     & 1            & 1            & I \textgreater 500  \\ \hline
\end{tabular}
\end{table}
\textit{Other Parameters: Defiance, Global Exposure, Commuter Ratio By Station, Awareness Increase Rate, Borough Commuting Modifier...}
\end{frame}
\begin{frame}
\frametitle{Compartmental Modeling Results (NYC)}
\begin{figure}
\includegraphics[width=0.8\textwidth]{Scratch_Visuals/SEIR_Curve_NYC_3.png}
\caption{SEIR Fitting to NYC Case Total. \textit{MAPE ($t \geq 30$): 0.0145}}
\end{figure}
\end{frame}
\begin{frame}
\frametitle{Results by MODZCTA (NYC)}
\begin{figure}[htbp]
\includegraphics[width=1.0\textwidth]{Scratch_Visuals/NYC_Geo_Fitting.png}
\begin{tiny}
Scale(0 - 0.11 infections/person, 0 - 0.11 infections/person, 0 - 0.20 cumulative cases/person) 
\end{tiny}
\end{figure}
\textit{MAPE ($t \geq 32, Stations \geq 1 = 0.311$ )}\\
\href{https://github.com/cheung-ho-lum/NS_Epidemics_ABM_Approach/blob/master/Repository/Visualizations/infection_timelapse_NYC_3.gif}{New York City Timelapse}
\end{frame}
\begin{frame}
\frametitle{Discussion}
\begin{itemize}
    \item More granular passenger flow and population data.
    \item New York City is not a closed system.
    \item Control for population density, income, and other factors.
    \item Simplify away the feature bloat
\end{itemize}
\end{frame}
\begin{frame}
\frametitle{Conclusions}
\textit{All models are wrong, but some are useful -George Box}\\~\\
\begin{itemize}
    \item Determining passenger flow is a difficult problem for all transportation networks at all granularities.
    \item Rush hour and hubs contribute strongly to disease propagation
    \item Commute time is correlated to prevalence of ILIs
    \item Subway routes are correlated to prevalence of ILIs
\end{itemize}
\end{frame}
\begin{frame}
\frametitle{Acknowledgements (and Questions)}
Frank Acquaye - WAN, Passenger Flow\\
Ho Lum Cheung - NYC, Organization, Research, Testing\\
Dimas Muñoz-Montesinos - Madrid, Hotspots\\
Elie Wanko - Theory, Consulting\\
\href{https://github.com/cheung-ho-lum/NS_Epidemics_ABM_Approach}{Project on Github}
\end{frame}
\begin{frame}[shrink=50]
\frametitle{References}
\bibliography{bib_file}
\bibliographystyle{unsrt}
\end{frame}
\end{document}